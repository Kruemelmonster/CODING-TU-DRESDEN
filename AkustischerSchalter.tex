\documentclass[a4paper,12pt]{article}
\usepackage[ngerman]{babel}   % german layout, new german orthography
\usepackage[utf8x]{inputenc} % direct use of german umlaute
%\usepackage[latin1,utf8]{inputenc} % direct use of german umlaute, utf8 encoding
\usepackage[T1]{fontenc}      % correct german hyphenation of words with umlaute
\pagestyle{plain}             % layout with plain page number
%\pagestyle{empty}             % empty heads and feet, no page number
\parindent 0pt                % no paragraph indentation

\usepackage[a4paper,top=25mm,left=20mm,right=20mm,bottom=20mm]{geometry} % modify page geometry
%\usepackage{a4wide}    % paper format DIN A4
%\usepackage{fancyhdr}  % page layout with header and footer
\usepackage{hyperref}  % document navigation with hyperlinks
\usepackage{graphicx}  % graphical elements
%\usepackage{pstricks}  % postscript drawings
\usepackage{textcomp}  % additional symbols
%\usepackage{units}     % setting units typographically correct
%\usepackage{color}     % colored elements
\usepackage{xcolor}    % colored elements, extended package

\usepackage{amsmath}   % math formulae
\usepackage{amssymb}   % math symbols
\usepackage{amsfonts}  % additional math fonts
%\usepackage{enumerate} % enumerated itemization
\usepackage{verbatim}  % verbatim display/commenting pieces of LaTeX code
\usepackage{listings}  % display of source code listings
%\usepackage{url}       % weblinks and e-mail addresses
\usepackage{textcomp}
%\pagestyle{fancy} % page layout with header and footer, requires fancyhdr-package
%\fancyhf{}
%\fancyhead[L]{\footnotesize \leftmark}
%\fancyhead[R]{\footnotesize \rightmark}
%\fancyfoot[C]{\small \thepage}
\usepackage{url}
\hypersetup{
%    bookmarks=false,               % show bookmarks bar?
    unicode=false,                 % non-Latin characters in Acrobat's bookmarks
    pdftoolbar=true,               % show Acrobat's toolbar?
    pdfmenubar=true,               % show Acrobat's menu?
    pdffitwindow=false,            % window fit to page when opened
    pdfstartview={FitH},           % fits the width of the page to the window
    pdftitle={MRT1-V3},            % title
    pdfauthor={Gr.xx/WS2011},      % author
    pdfsubject={MRT1-Praktikum},   % subject of the document
    pdfcreator={},                 % creator of the document
    pdfproducer={},                % producer of the document
    pdfkeywords={C-Programmierung},% list of keywords
    pdfnewwindow=true,             % links in new window
    colorlinks=true,               % false: boxed links; true: colored links
    linkcolor=black,               % color of internal links
    citecolor=green,               % color of links to bibliography
    filecolor=red,                 % color of file links
    urlcolor=blue                  % color of external links
}


\definecolor{listinggray}{gray}{0.9}
\definecolor{lbcolor}{rgb}{0.9,0.9,0.9}
\lstset{
backgroundcolor=\color{lbcolor},
    tabsize=4,
%   rulecolor=,
    language=[GNU]C++,
        basicstyle=\scriptsize,
        upquote=true,
        aboveskip={1.5\baselineskip},
        columns=fixed,
        showstringspaces=false,
        extendedchars=false,
        breaklines=true,
        prebreak = \raisebox{0ex}[0ex][0ex]{\ensuremath{\hookleftarrow}},
        frame=single,
        numbers=left,
        showtabs=false,
        showspaces=false,
        showstringspaces=false,
        identifierstyle=\ttfamily,
        keywordstyle=\color[rgb]{0,0,1},
        commentstyle=\color[rgb]{0.026,0.112,0.095},
        stringstyle=\color[rgb]{0.627,0.126,0.941},
        numberstyle=\color[rgb]{0.205, 0.142, 0.73},
%        \lstdefinestyle{C++}{language=C++,style=numbers}’.
}


\begin{document}

\thispagestyle{empty}

\begin{center}

~\\ \vspace{100pt}

\underline{\hspace{300pt}}\\~\\~\\
{\bf \large Mikrorechentechnik II- } {\large \bf Akustischer Schalter}\\~\\
\underline{\hspace{300pt}}\\

\vspace{50pt}
Praktikum Mikrorechentechnik II\\
im Sommersemester 2013\\~\\
04.Juli.2013
\vspace{200pt}

Versuchsprotokoll\\~\\
Gruppe 27\\~\\
{\it Bastian Balk (2749649)\\ Timo Fuckner (3764171)\\ Mantvydas Kalibatas (3829982)\\ Arne Klaß (3805971)}\\~\\

Betreuer: Dipl.-Ing. Stephan Hübler 

\end{center}

\newpage

\thispagestyle{empty}

~\\ \vspace{40pt}

\tableofcontents
\newpage
\section{Bearbeitung der Aufgabenstellung}
Die Aufgabe konnte anhang der zur Verfügung gestellen Aufgabenstellung größtenteils ohne Probleme bewerkstelligt werden. Insgesamt mussten 3 Funktionen geschrieben werden: Signaldetektion,Signalanalyse 
mit Berechnung der Nulldurchgangsdichte und Berechnung des Nulldurchgangshistogrammes. 
\subsection{Signaldetektion}
\begin{figure}[!htp]
\centering
\includegraphics[]{./detec.pdf}
\end{figure}

Hierbei sollte der relevante Signalabschnitt ermittelt werden. Dies geschah mittels der Berechnung des Energieverlaufs. Wenn dieser einen gewissen Schwellwert, siehe Abbildung\ref{Detektion}, überschritten hat wurde dieser Teil extrahiert
und zur weiteren Verarbeitung verwendet.

Ein interessantes Problem ergabs sich hierbei bei der Speicherung der Effektivwerte in dem Array.
\begin{lstlisting}[caption={detection.c}]
long int eVerlauf[Messwertanzahl/FENSTER];            //Energieverläufe werden hier gespeichert

for(i= 0; i <= *sample_anzahl; i++) {
	teil += (*(*sample_anfang+i))*(*(*sample_anfang+i));        //berechnen des quadratischen effektivwertes
	if( i>1 && i % FENSTER == 0 ) {
		if(counter>(Messwertanzahl/FENSTER)) {
				printf("error1");                                   
				return -1;}
			eVerlauf[counter] = teil;
			counter++;
			teil=0;}
	}
 
\end{lstlisting}
Da mit einer 32 Bit Maschine gearbeitet wurde, stellte long int nur einen Wertebereich von 2$^31$ \^= 2,147483648*10$^9$ bereit.
In einem Extremfall hätte sich als Effektivwert 32000$^2$ \^= 1.024$^9$ ergeben. Somit hätte man schon bei 2 Werten einen Überlauf erhalten.
Um dieses Problem zu beheben muss man den Effektivwert bevor er in den Array gespeichert wird auf die Messwertanzahl normieren.



\begin{figure}[!htp]
\centering
\includegraphics[scale=0.6]{./Detektion.pdf}
\caption{Signaldetektion}
\label{Detektion}

\end{figure}
\newpage

\subsection{Nulldurchgangsdichte}
\begin{figure}[!htp]
\centering
\includegraphics[]{./Dichte.pdf}
\end{figure}
\begin{lstlisting}[caption={dichte.c}]
int ndg_dichte(short *feld_ptr, unsigned int anzahl_atw, float *dichte_ori, float *dichte_diff){

   int i = 0;

	int ndg = 0;      //Initialisierung von Zähler(i) und Nulldurchgänge
	short fall = false, below = false, above =false; //Initialisierung von "Merkmale" für NDG

  *dichte_ori = (float)0.0;

  *dichte_diff = (float)0.0;

	for(i;i<anzahl_atw;i++) {             //Alle Abtastwerte(atw) abarbeiten
		if(*(feld_ptr+i) >= s)
			above=true;
		if(*(feld_ptr+i) <= -s)
			below=true;
		if(above & !below) 
			fall =true;            
		if(!above  && below)
			fall =false;
		if(above  && below ){       //Sind above und below gesetzt fand ein NDG statt
			ndg++;

			if (fall)
				above = false ;
			else
				below = false;}
	}
	printf("ndg=%i",ndg);
	*dichte_ori =anzahl_atw/ndg; /*dichte_ori=Anzahl der Nulldurchgänge im vorgegebenen
				       Abtastwertefeld im Verhältnis zur Anzahl der Abtastwerte im gesamten Analysefenster.*/

\end{lstlisting}

Diese Funktion funktionierte von vornherein ohne Probleme. Die größtee Schwierigkeiten ergab sich hierbei durch das richtige zurücksetzen der Variablen above und below.
Dies wurde jedoch durch die Hilfsvariable Fall sehr gut bewerkstelligt.

\subsection{Nulldurchgangshistogramm}
Diese Funktion konnte leider nicht vollständig fertiggestellt werden. Was auch die vielen print Ausgaben erklärt, die zum Debuggen gediehnt haben.
Außerdem ergab sich seltsamerweiße ein Problem bei der doppelten Verwendung von Variablen. Obwohl diese bei einer erneuten Benutzung jedesmal neu mit 0 initialisiert wurden, ergab sich der 
gewünschte Effekt erst nachdem eine neue Variable dafür eingesetzt wurde.


\newpage
\section{Quelltext}
\begin{lstlisting}[caption={detection.c}]
/* Detektion des Signals */

#include <stdio.h>

#include <stdlib.h>

#define FENSTER 320        /* Fensterlaenge=20 ms fuer Energieberechnung */

#define MIN_FENSTER_ANZ 10 /* Minimal 10 Fenster Signal */

#define ABSTAND 10        /* Abstand Effektivwertquadrat Signal-Pause */

#define Messwertanzahl 32000 //Anleitung

int signal_detekt(short **sample_anfang,unsigned int *sample_anzahl)

{

int counter = 0, max = 0, threshold = 0, time = 0, p=0, hoehstes = 0, anfang = 0;

	int  i;

	long int teil = 0;

	long int eVerlauf[Messwertanzahl/FENSTER];            //Energieverläufe werden hier gespeichert


	for(i= 0; i <= *sample_anzahl; i++) {

		teil += (*(*sample_anfang+i))*(*(*sample_anfang+i));        //berechnen des quadratischen effektivwertes

		if( i>1 && i % FENSTER == 0 ) {				    // Über die Fensterlänge iterieren 		

			if(counter>(Messwertanzahl/FENSTER)) {              // Falls 100 Überschritten wird "error1" 

				printf("error1");                                   

				return -1;}

			eVerlauf[counter] = teil;                           //Speicher Effektivwerte in Array

			counter++;

			teil=0;}

	}

	for(i=0;i<=Messwertanzahl/FENSTER;i++){

		if(max<eVerlauf[i])   

			max=eVerlauf[i];}  //Höchster Wert als Max gespeichert





	threshold=max/(Messwertanzahl/FENSTER);        // Schwelle festsetzen

	// Finden des geeingeten Abschnitts

	for(i=0;i<=Messwertanzahl/FENSTER;i++){

		if(eVerlauf[i]>= threshold){   // Effektivwerte die über der Schwelle liegen

			p=i;

			time = 0;

			while(eVerlauf[p]>=threshold && p< (Messwertanzahl/FENSTER)){

				p++;

				time++;}



			if(hoehstes < time){

				hoehstes=time;  //Neue Höchstzeit gefunden

				anfang = i;     //Anfang ist der Beginn des Nutzsignals 

			}



		}

	}

	//Wenn Mindestzeit nicht erreicht wurde muss 1 zurückgegeben werden



	if(hoehstes<MIN_FENSTER_ANZ)

		return 1;



	*sample_anfang += anfang*FENSTER;

	*sample_anzahl =hoehstes*FENSTER;

return (0);

}
\end{lstlisting}


\begin{lstlisting}[caption={dichte.c}]
/* Berechnung der NDG - Dichte eines Signalabschnittes */



#include <stdio.h>

#include <stdlib.h>

#include <math.h>

//#include <stdbool.h>

#define true 1

#define false 0



int s = 60;

int ndg_dichte(short *feld_ptr, unsigned int anzahl_atw, float *dichte_ori, float *dichte_diff){

   int i = 0;

	int ndg = 0;      //Initialisierung von Zähler(i) und Nulldurchgänge
	short fall = false, below = false, above =false; //Initialisierung von "Merkmale" für NDG

  *dichte_ori = (float)0.0;

  *dichte_diff = (float)0.0;

	for(i;i<anzahl_atw;i++) {             //Alle Abtastwerte(atw) abarbeiten
		if(*(feld_ptr+i) >= s)
			above=true;
		if(*(feld_ptr+i) <= -s)
			below=true;
		if(above & !below) 
			fall =true;                // Fall Hilfsvariable um je nachdem above oder below false zu setzen
		if(!above  && below)
			fall =false;
		if(above  && below ){       //Sind above und below gesetzt fand ein NDG statt
			ndg++;

			if (fall)
				above = false ;  // Falls  "gesetzt" ist, wird
			else
				below = false;}
	}
	printf("ndg=%i",ndg);
	*dichte_ori =anzahl_atw/ndg; /*dichte_ori=Anzahl der Nulldurchgänge im vorgegebenen
				       Abtastwertefeld im Verhältnis zur Anzahl der Abtastwerte im gesamten Analysefenster.*/


	i = 0; ndg = 0;
	above = false; fall = false; below = false; //Alle Werte resetten
	// Selber Code wie oben nur für differenziertes Signal

	for(i;i < anzahl_atw-1;i++) {// evtl. +1 !?!
		if(*(feld_ptr+i)-*(feld_ptr+i+1) >= s){
			above = true;}
		if(*(feld_ptr+i)-*(feld_ptr+i+1) <= -s){
			below = true;}
		if(above  && !below )
			fall = true;
		if(!above && below )
			fall = false;
		if(above && below ){
			ndg++;
			if(fall )
				above = false;
			else
				below = false;}

	}
	*dichte_diff = anzahl_atw/ndg;



return (0);

}

\end{lstlisting}
\begin{lstlisting}[caption={histo.c}]
#include <math.h>

#include <stdio.h>

#include <stdlib.h>



//#include <detection.c>

#define Messwertanzahl 32000

//benötigte angaben

#define true 1

#define false 0

int s=50;                      //schwelle für NGD





int ndg_histogramm(short *feld_ptr, unsigned int anzahl_atw, float *hist_ori, float *hist_diff){



	int p = 0;	//Abstand zwischen Nulldurchgängen

	int distance = 0,a=0; //a um p in feld zu speichern

	int feld[Messwertanzahl];

	int i = 0, ndg = 0, b=0, z=0, f=0, j=0;

	int mittel,grenze1,grenze2;

	unsigned int min =32000, max=0;

	char fall = false, above = false, across=false;



	for(p=0;p<4;p++){ 		//Nullsetzen alles Histogramme

		*(hist_ori+p)=0;

		*(hist_diff+p)=0;

	}



	for(i;i<anzahl_atw;i++) {    //genauso wie bei *Dichte*

		if(*(feld_ptr+i) >= s)

			across=true;

		if(*(feld_ptr+i) <= -s)

			above=true;

		if(across & !above)

			fall =true;

		if(!across  && above)

			fall =false;

		if(across  && above ){

      //printf(" p is %d | ",p);

      if(b){

			feld[a]=p;   //Speichert Abstand zwischen den NDG an der Stelle a in Array

			distance +=p;

         a++;

         }

         b=1;

			if (fall)

				across = false ;

			else

				above = false;

			}

		//wenn ein ndg stattfand muss p angeglichen werden
		//P wird über "echten" NDG angeglichen


		if((*feld_ptr+i)>=0 && *(feld_ptr+i+1) <0 || (*(feld_ptr+i)<= 0 && *(feld_ptr+i+1) >0)){

			p = i-distance;

         i++;              }

	}

}

for(f=0; f<a;f++) printf("p[%d)=%d | ",f,feld[f]);

//Restlichen Felder 0 setzen

while(a<= anzahl_atw){ //Setze restliche Felder 0

	feld[a]=0;

	a++;

}



//Bestimmung von Max und Min für Zuordnung

for(z=0; z<anzahl_atw;z++){

	if(feld[z]>max)

		max=feld[z];

	if((feld[z]<min) && feld[z] !=0)

		min=feld[z];

      }

	//Bestimmung der 4 Gruppen,

	mittel=(max+min)/2;

	grenze1=(mittel+min)/2;

	grenze2=(min+grenze1)/2;

   printf(" The limits are %d %d %d %d %d\n",min, mittel,grenze1,grenze2,max);

	//Nulldurchgänge den Gruppen zuorden



	for(j=0; j < anzahl_atw; j++) {

   printf("<FIRST!!!");

		if(min<=feld[j] && feld[j] < grenze1 ){

			(*hist_ori)++ ;

         }

		if(grenze1<=feld[j] && feld[j] < mittel )

			(*(hist_ori+1))++;

		if(mittel<=feld[j] && feld[j] < grenze2 )

			(*(hist_ori+2))++;

		if(grenze2<=feld[j] && feld[j] < max )

			(*(hist_ori+3))++;

		//Genauso wie oben

		 fall = false, above = false, across=false;

       a=0; i=0;



		for(i;i < anzahl_atw-1;i++) {	   
		  if(*(feld_ptr+i)-*(feld_ptr+i+1) >=  s)

				across = true;

			if(*(feld_ptr+i)-*(feld_ptr+i+1) <=  -s)

				above = true;

			if(across  && !above )

				fall = true;

			if(!across && above )

				fall = false;

			if(across && above ){

				feld[a]=p;

				distance +=p;



				if(fall )

					across = false;

				else

					above = false;

				a++;

			}



			//P anpassen über keinen "gezählten" NDG



			if((*(feld_ptr+i)-*(feld_ptr+i+1) >= 0 && *(feld_ptr+i+1)-*(feld_ptr+i+2) < 0) || (*(feld_ptr+i)-*(feld_ptr+i+1) <= 0 && *(feld_ptr+i+1)-*(feld_ptr+i+2) > 0)){

                           p = i-laenge;

			}

		}

		//Restlichen Felder 0 setzen

		while(a<= anzahl_atw){

			feld[a]=0;

			a++;

		}

		//Bestimmung von Max und Min für Zuordnung

		for(a=0; a<=anzahl_atw;a++){

			if(feld[a]>max)

				max=feld[a];



			if((feld[a]<min) && feld[a] !=0)

				min=feld[a];

		}



		//Bestimmung der 4 Gruppen,

		mittel=(max+min)/12;

		grenze1=(mittel+min)/3;

		grenze2=(mittel+max)/3;





		//Nulldurchgänge den Gruppen zuorden



		for(a=0; a <= anzahl_atw; a++) {

			if(min<=feld[a] && feld[a] < grenze1 )

				(*hist_diff)++ ;

			if(grenze1<=feld[a] && feld[a] < mittel )

				(*(hist_diff+1))++;

			if(mittel<=feld[a] && feld[a] < grenze2 )

				(*(hist_diff+2))++;

			if(grenze2<=feld[a] && feld[a] < max )

				(*(hist_diff+3))++;

		}

		return 0;

	}

   }
\end{lstlisting}


\end{document}
